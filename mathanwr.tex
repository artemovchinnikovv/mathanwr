\documentclass{article}
\usepackage{graphicx} % required for inserting images
\usepackage{amsmath,amsfonts,amssymb,amsthm,mathtools} % math
\usepackage[unicode, pdftex]{hyperref} % hyperlinks
\usepackage{amsfonts} % NZQRC

\begin{document}

\section{First problem: Indefinite integral}
a. Rational function. \\
b. Fractional rational function with transcendental part.

\subsection{Rational function}

Methods: indefinite coefficients, application under differential. \\
Warnings: dont forget when you have found the coefficients, substitute them into the equation to make sure that they are correct. \\
Example: \href{https://old.mipt.ru/education/chair/mathematics/exams/exams/2022-23/%D0%9C%D0%90%D0%98%D0%B8%D0%A0_%D0%92_23.pdf}{2022–2023}, first task (\href{https://old.mipt.ru/education/chair/mathematics/exams/exams/2022-23/%D0%9C%D0%90%D0%98%D0%B8%D0%A0_%D0%92_23%D0%BE%D1%82%D0%B2%D0%B5%D1%82%D1%8B.pdf}{answer}).

\begin{equation}
    \int \frac{x^2+2}{(x+1)(x^2-2x+3)}dx = ?
\end{equation}

\begin{equation*}
    \int \frac{x^2+2}{(x+1)(x^2-2x+3)}dx = \int \frac{A}{x+1}dx + \int \frac{Bx+C}{x^2-2x+3}dx
\end{equation*}

\begin{equation}
    x^2+2 = A(x^2-2x+3) + (Bx+C)(x+1)
\end{equation}

\begin{equation*}
    1. \; 2 = 3A + C
\end{equation*}

\begin{equation*}
    x. \; 0 = -2A + B + C
\end{equation*}

\begin{equation*}
    x^2. \; 1 = A + B
\end{equation*}

\begin{equation}
    A = \frac{1}{2}, \; B = \frac{1}{2}, \; C = \frac{1}{2}
\end{equation}

\begin{equation*}
    \int \frac{x^2+2}{(x+1)(x^2-2x+3)}dx = \int \frac{\frac{1}{2}}{x+1}dx + \int \frac{\frac{x}{2}+\frac{1}{2}}{x^2-2x+3}dx = \frac{1}{2}ln|x+1| + \frac{1}{2} \int \frac{x+1}{x^2-2x+3}dx
\end{equation*}

\begin{equation*}
    \frac{1}{2} \int \frac{x+1}{x^2-2x+3}dx = \frac{1}{2} \int \frac{x+1}{2x-2}d(ln|x^2-2x+3|) = \frac{1}{4} \int \frac{2x-2+2+2}{2x-2}d(ln|x^2-2x+3|)
\end{equation*}

\begin{equation*}
    \frac{1}{4} \int d(ln|x^2-2x+3|) + \int \frac{1}{x^2-2x+3}dx = \frac{1}{4} ln|x^2-2x+3| + \int \frac{1}{(x-1)^2+2}dx
\end{equation*}

\begin{equation*}
    \frac{1}{4} ln|x^2-2x+3| + \frac{1}{\sqrt{2}} \arctan (\frac{x-1}{\sqrt{2}}) + C
\end{equation*}

\begin{equation}
    \int \frac{x^2+2}{(x+1)(x^2-2x+3)}dx = \frac{1}{2}ln|x+1| + \frac{1}{4} ln|x^2-2x+3| + \frac{1}{\sqrt{2}} \arctan (\frac{x-1}{\sqrt{2}}) + C
\end{equation}

\subsection{Fractional rational function with transcendental part.}

Methods: variable replacement, differentiation, integration by parts, application under differential. \\
Example: \href{https://old.mipt.ru/education/chair/mathematics/exams/exams/2022-23/%D0%9C%D0%90%D0%98%D0%B8%D0%A0_%D0%92_23.pdf}{2022–2023}, first task (\href{https://old.mipt.ru/education/chair/mathematics/exams/exams/2022-23/%D0%9C%D0%90%D0%98%D0%B8%D0%A0_%D0%92_23%D0%BE%D1%82%D0%B2%D0%B5%D1%82%D1%8B.pdf}{answer}).

\begin{equation}
    \int \frac{x exp(x)}{\sqrt{exp(x)-1}}dx = ?
\end{equation}

\begin{equation}
    t = \sqrt{exp(x)-1}
\end{equation}

\begin{equation}
    t^2+1 = exp(x)
\end{equation}

\begin{equation}
    ln(t^2+1) = x
\end{equation}

\begin{equation*}
    dt = \frac{1}{2} \frac{exp(x)}{\sqrt{exp(x)-1}}dx = \frac{1}{2} \frac{t^2+1}{t}dx
\end{equation*}

\begin{equation}
    dx = \frac{2tdt}{1+t^2}
\end{equation}

\begin{equation*}
    \int \frac{x exp(x)}{\sqrt{exp(x)-1}}dx = \int \frac{(t^2+1)ln(t^2+1)}{t} \frac{2tdt}{1+t^2} = \int 2ln(1+t^2)dt
\end{equation*}

\begin{equation*}
    \int 2ln(1+t^2)dt = 2tln(1+t^2) - \int 2t d(ln(1+t^2)) = 2tln(1+t^2) - \int 2t \frac{2t}{1+t^2}dt
\end{equation*}

\begin{equation*}
    2tln(1+t^2) - \int \frac{4t^2}{1+t^2}dt = 2tln(1+t^2) - \int \frac{4t^2+4-4}{1+t^2}dt
\end{equation*}

\begin{equation*}
    - \int 4dt + \int \frac{4dt}{1+t^2} + 2tln(1+t^2) = 2tln(1+t^2) - 4t + 4 \arctan (t) + C
\end{equation*}

\begin{equation}
    \int \frac{x exp(x)}{\sqrt{exp(x)-1}}dx = 2tln(1+t^2) - 4t + 4 \arctan (t) + C
\end{equation}


\newpage
\section{Second problem: Differentials and Taylor's formula for functions of several variables }
Methods: partial derivatives, Taylor's multivariable formula. \\
Warnings: dont forget about function value at a point (you must add it to Taylor's formula as constant) and about additional multipliers (if you write first derivative at Taylor's formula you divide by 1, second derivative by 2, third derivative by 6 etc.). \\
Example: \href{https://old.mipt.ru/education/chair/mathematics/exams/exams/2022-23/%D0%9C%D0%90%D0%98%D0%B8%D0%A0_%D0%92_23.pdf}{2022–2023}, second task (\href{https://old.mipt.ru/education/chair/mathematics/exams/exams/2022-23/%D0%9C%D0%90%D0%98%D0%B8%D0%A0_%D0%92_23%D0%BE%D1%82%D0%B2%D0%B5%D1%82%D1%8B.pdf}{answer}).

\begin{equation}
    f(x,y) = \tan (x+sin(xy^2)), \; M=(\frac{\pi}{4},0), \; df=? \; df^2=? \; f = ? + o((x-\frac{\pi}{4})^2+y^2), \; x \rightarrow \frac{\pi}{4}, \; y \rightarrow 0
\end{equation}

\begin{equation*}
    \frac{\partial}{\partial x} f \bigg|_M = \frac{1+cos(xy^2)y^2}{cos^2(x+sin(xy^2))} \bigg|_M = 2
\end{equation*}

\begin{equation*}
    \frac{\partial}{\partial y} f \bigg|_M = \frac{cos(xy^2)x2y}{cos^2(x+sin(xy^2))} \bigg|_M = 0 
\end{equation*}

\begin{equation*}
    \frac{\partial}{\partial x} \frac{\partial}{\partial y} f \bigg|_M = y \cdot ... \bigg|_M = 0
\end{equation*}

\begin{equation*}
    \frac{\partial}{\partial y} \frac{\partial}{\partial x} f \bigg|_M = \frac{cos^2(x+sin(xy^2))(cos(xy^2)2y-y^2sin(xy^2)x2y) - }{cos^4(x+sin(xy^2))}
\end{equation*}
\begin{equation*}
    \frac{ - (1+cos(xy^2)y^2)2cos(x+sin(xy^2))(-sin(x+sin(xy^2)))(cos(xy^2)x2y)}{cos^4(x+sin(xy^2))} \bigg|_M = y \cdot ... \bigg|_M = 0
\end{equation*}

\begin{equation*}
    \frac{\partial}{\partial x} \frac{\partial}{\partial x} f \bigg|_M = \frac{(cos^2(x+sin(xy^2)))(-sin(xy^2)y^2y^2)-}{cos^4(x+sin(xy^2))}
\end{equation*}
\begin{equation*}
    \frac{-(1+cos(xy^2)y^2)(2cos(x+sin(xy^2))(-sin(x+sin(xy^2)))(1+...y)}{cos^4(x+sin(xy^2))} \bigg|_M = \frac{-1 \cdot 2 cos(\pi/4) (-sin(\pi/4))}{cos^4(\pi/4)} = 
\end{equation*}
\begin{equation*}
    = \frac{4\cdot 2}{2} = 4
\end{equation*}

\begin{equation*}
    \frac{\partial}{\partial y} \frac{\partial}{\partial y} f \bigg|_M = \pi
\end{equation*}

\begin{equation}
    df = 2dx, \; d^2f=4dx^2+\pi dy^2
\end{equation}

\begin{equation*}
    f(M) = 1
\end{equation*}

\begin{equation}
    f =1 + 2(x-\frac{\pi}{4}) + 2(x-\frac{\pi}{4})^2 + \frac{\pi}{2}y^2 + o((x-\frac{\pi}{4})^2+y^2), \; x \rightarrow \frac{\pi}{4}, \; y \rightarrow 0
\end{equation}


\newpage
\section{Third problem:  Length, area and volume calculation using definite integral}

\subsection{Length}
Method: use suitable formula and calculate definite integral. \\
Need to know: \hyperlink{3.1}{formulas for length, area and volume}, trigonometric formulas. \\
Warnings: accurate calculation. \\
Example: \href{https://old.mipt.ru/education/chair/mathematics/exams/exams/2022-23/%D0%9C%D0%90%D0%98%D0%B8%D0%A0_%D0%92_23.pdf}{2022–2023}, third task (\href{https://old.mipt.ru/education/chair/mathematics/exams/exams/2022-23/%D0%9C%D0%90%D0%98%D0%B8%D0%A0_%D0%92_23%D0%BE%D1%82%D0%B2%D0%B5%D1%82%D1%8B.pdf}{answer}).

\begin{equation}
    x = sin^3(\frac{t}{2}), \; x = cos^3(\frac{t}{2}), z = \frac{3}{4}(t+sin(t)), \; 0 \leq t \leq \pi, \; |L|=?
\end{equation}

\begin{equation}
    |L| = \int_{a}^{b} \sqrt{(\frac{dx}{dt})^2+(\frac{dy}{dt})^2+(\frac{dz}{dt})^2}dt
\end{equation}

\begin{equation*}
    \frac{dx}{dt} = 3sin^2(t/2) cos(t/2) \frac{1}{2}
\end{equation*}

\begin{equation*}
    \frac{dy}{dt} = 3cos^2(t/2) (-sin(t/2)) \frac{1}{2}
\end{equation*}

\begin{equation*}
    \frac{dz}{dt} = \frac{3}{4}(1+cos(t))
\end{equation*}

\begin{equation*}
    |L| = \int_{0}^{\pi} \sqrt{\frac{9}{4} sin^2(t/2) cos^2(t/2) (sin^2(t/2)+cos^2(t/2))+\frac{9}{16}+\frac{9}{16}cos^2(t)+\frac{9}{8}cos(t)}dt
\end{equation*}

\begin{equation*}
    |L| = \int_{0}^{\pi} \sqrt{\frac{9}{4} sin^2(t/2) cos^2(t/2)+\frac{9}{16}+\frac{9}{16}cos^2(t)+\frac{9}{8}cos(t)}dt
\end{equation*}

\begin{equation*}
    |L| = \int_{0}^{\pi} \sqrt{\frac{9}{16} sin^2(t)+\frac{9}{16}+\frac{9}{16}cos^2(t)+\frac{9}{8}cos(t)}dt
\end{equation*}

\begin{equation*}
    |L| = \int_{0}^{\pi} \sqrt{\frac{9}{8}+\frac{9}{8}cos(t)}dt
\end{equation*}

\begin{equation*}
    |L| = \int_{0}^{\pi} \frac{3}{2\sqrt{2}} \sqrt{1+cos(t)}dt
\end{equation*}

\begin{equation*}
    cos^2(t/2) = \frac{1+cos(t)}{2}
\end{equation*}

\begin{equation*}
    |L| = \int_{0}^{\pi} \frac{3}{2} | cos(t/2) | dt
\end{equation*}

\begin{equation*}
    |L| = \int_{0}^{\pi} 3 | cos(t/2) | dt/2
\end{equation*}

\begin{equation*}
    |L| = \int_{0}^{\pi/2} 3 | cos(\phi) | d\phi
\end{equation*}

\begin{equation}
    |L| = 3
\end{equation}


\newpage
\section{Fifth problem: Improper integral of a function of constant sign with parameter}
Method: to divide integral to two integrals from 0 to 1 and from 1 to infinity, then find using equal functions at what value each integral converges, then combine results. \\
Need to know: equal functions. \\
Warnings: check that integral is constant sign and write it on the paper, because equal functions work only with constant sign functions. \\
Example: \href{https://old.mipt.ru/education/chair/mathematics/exams/exams/2022-23/%D0%9C%D0%90%D0%98%D0%B8%D0%A0_%D0%92_23.pdf}{2022–2023}, fifth task (\href{https://old.mipt.ru/education/chair/mathematics/exams/exams/2022-23/%D0%9C%D0%90%D0%98%D0%B8%D0%A0_%D0%92_23%D0%BE%D1%82%D0%B2%D0%B5%D1%82%D1%8B.pdf}{answer}).
\begin{equation}
    \int_{0}^{\infty} \frac{(\sqrt{x^2+x^4}-x)^\alpha }{(x- \ln x)^2 \ln (e^x-x)} dx
\end{equation}
\begin{equation*}
    let \; x \rightarrow 0, \; then
\end{equation*}
\begin{equation*}
    f(x) \sim \frac{(x \sqrt{1+x^2}-x)^\alpha}{\ln^2 x \ln (1+x+\frac{x^2}{2}-x)} \sim \frac{(x+\frac{x^3}{2}-x)^\alpha}{\ln^2 x \cdot x^2} \sim \frac{x^{3\alpha}}{\ln^2 x \cdot x^2} \sim \frac{1}{\ln^2 x \cdot x^{2-3\alpha}}
\end{equation*}
\begin{equation*}
    congerges, \; if \; (2-3\alpha)<1, \; but \; we \; have \; \ln ^ 2 x, \; so
\end{equation*}
\begin{equation*}
    (2-3\alpha) \leq 1
\end{equation*}
\begin{equation*}
    1 \leq 3 \alpha
\end{equation*}
\begin{equation}
    \frac{1}{3} \leq \alpha
\end{equation}
\begin{equation*}
    let \; x \rightarrow \infty, \; then
\end{equation*}
\begin{equation*}
    f(x) \sim \frac{x^{2 \alpha}}{x^2 \cdot x} \sim \frac{1}{x^{3-2\alpha}}
\end{equation*}
\begin{equation*}
    congerges, \; if \; (3-2\alpha)>1
\end{equation*}
\begin{equation*}
    2 > 2 \alpha
\end{equation*}
\begin{equation}
    1 > \alpha
\end{equation}
\begin{equation*}
    finally
\end{equation*}
\begin{equation}
    \int_{0}^{\infty} \frac{(\sqrt{x^2+x^4}-x)^\alpha }{(x- \ln x)^2 \ln (e^x-x)} dx \; converges \; at \; \frac{1}{3} \leq \alpha < 1
\end{equation}


\newpage
\section{Sixth problem: Convergence of a constant sign number series}
Method: to find a suitable convergence test.\\
Need to know: \hyperlink{6.1}{d'Alembert's convergence test}, Cauchy's convergence test.\\
Warnings: most convergence tests works only with constant sign number series.\\
Example: \href{https://old.mipt.ru/education/chair/mathematics/exams/exams/2022-23/%D0%9C%D0%90%D0%98%D0%B8%D0%A0_%D0%92_23.pdf}{2022–2023}, sixth task (\href{https://old.mipt.ru/education/chair/mathematics/exams/exams/2022-23/%D0%9C%D0%90%D0%98%D0%B8%D0%A0_%D0%92_23%D0%BE%D1%82%D0%B2%D0%B5%D1%82%D1%8B.pdf}{answer}).

\begin{equation}
    \sum_{n=1}^{\infty} \frac{C_{3n}^n}{7^n}
\end{equation}

\begin{equation*}
    \underset{n \rightarrow \infty}{lim} \frac{C_{3n+3}^{n+1}}{7^{n+1}} \frac{7^n}{C_{3n}^{n}} = \underset{n \Rightarrow \infty}{lim} \frac{1}{7} \frac{(3n+3)!}{(n+1)!((3n+3)-(n+1))!} \frac{n!(3n-n)!}{3n!}
\end{equation*}

\begin{equation*}
    \underset{n \rightarrow \infty}{lim} \frac{1}{7} \frac{(3n+1)(3n+2)(3n+3)}{(n+1)(2n+1)(2n+2)} = \frac{3 \cdot 3 \cdot 3}{7 \cdot 2 \cdot 2} = \frac{27}{28} < 1
\end{equation*}

     $\Rightarrow$ number series converges using d'Alembert's ratio test.


\newpage
\section{Seventh problem: Pointwise and uniform convergence of functional sequence and functional series}
a. Functional sequence \\
b. Functional series
Method: find a limit, limit functional sequence / functional series on a set or prove that functional sequence can dis-converge to this limit.\\
Need to know: equal functions, Taylor's formula. \\
Warnings: if you prove limit it doesn't say anything about uniform convergence, you must limit it, and don't forget that equal functions work only with sign constant functions. \\
Example: \href{https://old.mipt.ru/education/chair/mathematics/exams/exams/2022-23/%D0%9C%D0%90%D0%98%D0%B8%D0%A0_%D0%92_23.pdf}{2022–2023}, seven task (\href{https://old.mipt.ru/education/chair/mathematics/exams/exams/2022-23/%D0%9C%D0%90%D0%98%D0%B8%D0%A0_%D0%92_23%D0%BE%D1%82%D0%B2%D0%B5%D1%82%D1%8B.pdf}{answer}).
\subsection{Functional sequence}
\begin{equation}
    E_1=(0,1), \; E_2=(1, +\infty), \; f_n (x) = \frac{n}{x} \sin \frac{x^2}{n}
\end{equation}
\begin{equation*}
    \underset{n \rightarrow \infty}{lim} f_n (x) = \underset{n \rightarrow \infty}{lim} \frac{n}{x}(\frac{x^2}{n}-\frac{x^6}{6n^3}+o(\frac{1}{n^3})) = \underset{n \rightarrow \infty}{lim} x + o(\frac{1}{n}) = x
\end{equation*}
\begin{equation*}
    \Rightarrow \; pointwise \; converges \; on \; the \; sets \; E_1 \; and \; E_2
\end{equation*}
\begin{equation*}
    lets \; look \; on \; E_1
\end{equation*}
\begin{equation*}
    |\frac{n}{x} \sin \frac{x^2}{n} - x| = |\frac{n}{x}(\sin \frac{x^2}{n}-\frac{x^2}{n})| \leq |\frac{n}{x}\frac{x^4}{2n^2}| \leq \frac{1}{2n} \rightarrow 0 \; (biggest \; x=1)
\end{equation*}
\begin{equation}
    \Rightarrow \; uniform \; converges \; on \; the \; set \; E_1
\end{equation}
\begin{equation*}
    lets \; look \; on \; E_2
\end{equation*}
\begin{equation*}
    let \; x = \sqrt{n}
\end{equation*}
\begin{equation}
    \frac{n}{x} \sin \frac{x^2}{n} = \sqrt{n} \sin 1 \rightarrow \infty \; at \; n \rightarrow \infty \Rightarrow \; uniform \; disconverges \; on \; the \; set \; E_2
\end{equation}

\newpage
\section{Eighth problem: Taylor series}
Method: calculate derivative of a complex part, expand the result in Taylor series, integrate result, multiply result to not complex part. \\
Need to know: \hyperlink{8.1}{derivative table (trigonometric)}, \hyperlink{8.2}{Taylor series table}, \hyperlink{8.3}{Cauchy–Hadamard theorem}. \\
Warnings: don't forget to correctly calculate derivative of complex function (that you must multiply derivative of argument), don't forget that you must add constant after integration (function value in the null), if you have x to the power 2n in your series, you must take square root of R that you calculated with Cauchy–Hadamard theorem \\
Example: \href{https://old.mipt.ru/education/chair/mathematics/exams/exams/2022-23/%D0%9C%D0%90%D0%98%D0%B8%D0%A0_%D0%92_23.pdf}{2022–2023}, eight task (\href{https://old.mipt.ru/education/chair/mathematics/exams/exams/2022-23/%D0%9C%D0%90%D0%98%D0%B8%D0%A0_%D0%92_23%D0%BE%D1%82%D0%B2%D0%B5%D1%82%D1%8B.pdf}{answer}).

\begin{equation}
    f = x^2 \arccos \frac{\sqrt{2+x}}{2} = \sum ..?
\end{equation}

\begin{equation*}
    \frac{d}{dx} \arccos \frac{\sqrt{2+x}}{2} = \frac{-1}{\sqrt{1-\frac{2+x}{4}}} \frac{1}{4\sqrt{2+x}} = \frac{-1}{2\sqrt{4-2-x}\sqrt{2+x}} = \frac{-1}{2\sqrt{4-x^2}}
\end{equation*}

\begin{equation*}
    \frac{-1}{2\sqrt{4-x^2}} = -\frac{1}{4}\sum_{k=0}^{\infty} C_{-1/2}^{k} (\frac{-x^2}{4})^{n} = \sum_{k=0}^{\infty} C_{-1/2}^{k} (-1)^{n+1} \frac{x^{2n}}{4^{n+1}}
\end{equation*}

\begin{equation*}
    \arccos \frac{\sqrt{2+x}}{2} = \arccos (\frac{\sqrt{2}}{2}) + \sum_{k=0}^{\infty} C_{-1/2}^{k} (-1)^{n+1} \frac{x^{2n+1}}{4^{n+1} (2n+1)}
\end{equation*}

\begin{equation}
    f = \frac{\pi}{4} x^2 + \sum_{k=0}^{\infty} C_{-1/2}^{k} (-1)^{n+1} \frac{x^{2n+3}}{4^{n+1} (2n+1)}
\end{equation}

\begin{equation*}
    \underset{n \rightarrow \infty}{lim} | \frac{C_{-1/2}^{n}}{4^{n+1}(2n+1)} \frac{4^{n+2}(2n+3)}{C_{-1/2}^{n+1}}| = \underset{n \rightarrow \infty}{lim} | 4 \frac{n+1}{-1/2-n} | = 4
\end{equation*}

\begin{equation*}
    4 \sim x^{2n} \Rightarrow 2 \sim x^n
\end{equation*}

\begin{equation}
    R = 2
\end{equation}


\newpage
\section{Third problem: Theory}

\hypertarget{3.1}{}
\subsection{Formulas for length, area and volume}
Length $|L|$:
\begin{equation*}
    |L| = \int_a^b \sqrt{(\frac{dx}{dt})^2+(\frac{dy}{dt})^2+(\frac{dz}{dt})^2}dt
\end{equation*}
Area $|S|$ rotating around the $Ox$ axis:
\begin{equation*}
    |S| = 2 \pi \int_a^b y \sqrt{(\frac{dx}{dt})^2+(\frac{dy}{dt})^2}dt
\end{equation*}
Area $|S|$ rotating around the $Oy$ axis:
\begin{equation*}
    |S| = 2 \pi \int_a^b x \sqrt{(\frac{dx}{dt})^2+(\frac{dy}{dt})^2}dt
\end{equation*}
Volume $|V|$ rotating around the $Ox$ axis:
\begin{equation*}
    |V| = \pi \int_a^b y^2 dx
\end{equation*}
Volume $|V|$ rotating around the $Oy$ axis:
\begin{equation*}
    |V| = \pi \int_a^b x^2 dy
\end{equation*}


\newpage
\section{Sixth problem: Theory}

\hypertarget{6.1}{}
\subsection{d'Alembert's convergence test}
\begin{equation*}
    1. \; let \; a_n>0 \; n \in \mathbb N
\end{equation*}
\begin{equation*}
    2. \; let \; \underset{n \rightarrow \infty}{lim} \frac{a_{n+1}}{a_n} = \lambda \; exist
\end{equation*}
\begin{equation*}
    then \; if
\end{equation*}
\begin{equation*}
    q<1 \; \sum_{n=0}^{\infty} a_n \; converges
\end{equation*}
\begin{equation*}
    q>1 \; \sum_{n=0}^{\infty} a_n \; diverges
\end{equation*}
\begin{equation*}
    q=1 \; \sum_{n=0}^{\infty} a_n \; can \; converge \; or \; diverge
\end{equation*}


\newpage
\section{Eighth problem: Theory}

\hypertarget{8.1}{}
\subsection{Trigonometric derivative table}
\begin{equation*}
    \frac{d}{dx}\arcsin x = \frac{1}{\sqrt{1-x^2}}
\end{equation*}
\begin{equation*}
    \frac{d}{dx}\arccos x = \frac{-1}{\sqrt{1-x^2}}
\end{equation*}
\begin{equation*}
    \frac{d}{dx} \arctan x = \frac{1}{1+x^2}
\end{equation*}
\begin{equation*}
    \frac{d}{dx} arcctg x = \frac{-1}{1+x^2}
\end{equation*}

\hypertarget{8.2}{}
\subsection{Taylor series table}
\begin{equation*}
    (1+x)^\alpha = \sum_{n=0}^{\infty} C_{\alpha}^n x^n
\end{equation*}
\begin{equation*}
    C_{\alpha}^n = \frac{\alpha(\alpha - 1)...(\alpha - (n - 1))}{n!}
\end{equation*}

\hypertarget{8.3}{}
\subsection{Cauchy–Hadamard theorem}
\begin{equation*}
    R = \underset{n \rightarrow \infty}{lim} |\frac{c_n}{c_{n+1}}|
\end{equation*}

\end{document}
